\documentclass[a4paper]{article}

%--------------------------------------------------
% Usefull packages
%--------------------------------------------------
\usepackage{mathtools}
\usepackage{graphicx}
\usepackage{listings}
\usepackage{optidef}
\usepackage{texlogos}
\usepackage{commath}
\usepackage{algpseudocode}
\usepackage{amsthm}
\usepackage{amsfonts}
\usepackage{bbold}


\makeatletter
\renewcommand*\env@matrix[1][*\c@MaxMatrixCols c]{%
  \hskip -\arraycolsep
  \let\@ifnextchar\new@ifnextchar
  \array{#1}}
\makeatother

\newtheorem{definition}{Definition}

\newcommand{\Reals}{\mathbb{R}}
\newcommand{\I}{\mathbb{1}}
\newcommand{\ZeroMat}{\mathbb{0}}

\newcommand{\tmpc}{\texttt{tmpc}}
\newcommand{\CasADi}{\texttt{CasADi}}
\newcommand{\HPMPC}{\texttt{HPMPC}} 
\newcommand{\HPIPM}{\texttt{HPIPM}}
\newcommand{\qpOASES}{\texttt{qpOASES}}
\newcommand{\MATLAB}{\texttt{MATLAB}}
\newcommand{\Eigen}{\texttt{Eigen3}}
\newcommand{\Blaze}{\texttt{Blaze}}

\newcommand{\bbmat}{\begin{bmatrix}}
\newcommand{\ebmat}{\end{bmatrix}}

\def\doubleunderline#1{\underline{\underline{#1}}}
\def\doubleoverline#1{\overline{\overline{#1}}}
%--------------------------------------------------
% TITLE
%--------------------------------------------------

% NOTE: what's in [ ] goes to footer for the fields title, author and date
\title{A \cpluspluslogo{} Library for Model Predictive Control}
%\subtitle{\large Presentation subtitle}
\author{Mikhail Katliar}
\date{\today}

\begin{document}

\maketitle

\section{Introduction to \tmpc{}}

\section{Quadratic Programming}

\begin{definition}[OCP QP]
	An OCP QP is a QP in the form\footnote{Slightly changed Gianluca's notation; $x$ comes before $u$.}
	\begin{mini}|l|[0]
		{\substack{x_0,x_1,\dots,x_{N-1}\\u_0,u_1,\dots,u_{N-1}}}{\sum_{n=0}^{N-1} \frac 1 2 \bbmat x_k \\ u_k \\ 1 \ebmat^\top \bbmat Q_k & S_k & q_k \\ S_k^\top & R_k & r_k \\ q_k^\top & r_k^\top & 0 \ebmat \bbmat x_k \\ u_k \\ 1 \ebmat}
		{}{}
		\addConstraint{x_{k+1}}{ = A_k x_k + B_k u_k + b_k,}{\quad k=0,\dots,N-1}
		\addConstraint{\bbmat \underline{x}_k \\ \underline{u}_k \ebmat}{\leq \bbmat x_k \\ u_k \ebmat \leq \bbmat \overline{x}_k \\ \overline{u}_k \ebmat,}{\quad k=0,\dots,N-1}
		\addConstraint{\underline{d}_k}{\leq \bbmat C_k & D_k \ebmat\bbmat x_k \\ u_k \ebmat \leq \overline{d}_k,}{\quad k=0,\dots,N-1}
		\label{eq:OCP_QP}
	\end{mini}
	where $u_k$ are the control inputs, $x_k$ are the states.
\end{definition}
An OCP QP consists of 16 elements: $Q,\,R,\,S,\,q,\,r,\,A,\,B,\,b,\,\underline{x},\,\underline{u},\,\overline{x},\,\overline{u},\,C,\,D,\,\underline{d},\,\overline{d}$.
Each of the elements has a time index $k$ which runs from $0$ to $N-1$. 
Note that $x_0,x_1,\dots,x_{N-1}$ are optimization variables, but $x_N$ is not.
It is convenient to group the elements corresponding to the same time index $k$ together:
\begin{definition}[QP stage]
	An OCP QP \emph{stage} is a combination of OCP QP elements corresponding to the same time index $k$:
	$$
	\mathcal{S}_k=(Q_k,R_k,S_k,q_k,r_k,A_k,B_k,b_k,\underline{x}_k,\underline{u}_k,\overline{x}_k,\overline{u}_k,C_k,D_k,\underline{d}_k,\overline{d}_k)\ .
	$$
\end{definition}

The problem \eqref{eq:OCP_QP} can be written in matrix form as
\begin{mini}|l|[0]
	{
		x,u
	}
	{
		\frac 1 2 \bbmat x \\ u \\ 1 \ebmat^\top 
		\bbmat Q & S & q \\ S^\top & R & r \\ q^\top & r^\top & 0 \ebmat 
		\bbmat x \\ u \\ 1 \ebmat
	}
	{}{}
	\addConstraint{A \bbmat x \\ x_N \ebmat} {= B u + b} {}
	\addConstraint
	{
		\bbmat \underline{x} \\ \underline{u} \ebmat
	}
	{
		\leq \bbmat x \\ u \ebmat \leq \bbmat \overline{x} \\ \overline{u} \ebmat
	}
	{}
	\addConstraint
	{
		\underline{d}
	}
	{
		\leq \bbmat C & D \ebmat\bbmat x \\ u \ebmat \leq \overline{d},
	}
	{}
	\label{eq:OCP_QP_matrix_form}
\end{mini}
where
\begin{multline}
	Q = \bbmat Q_0 \\ & \ddots \\ && Q_{N-1} \ebmat,
	\quad R = \bbmat R_0 \\ & \ddots \\ && R_{N-1} \ebmat,\\
	S = \bbmat S_0 \\ & \ddots \\ && S_{N-1} \ebmat,
	\quad q = \bbmat q_0 \\ \vdots \\ q_{N-1} \ebmat,
	\quad r = \bbmat r_0 \\ \vdots \\ r_{N-1} \ebmat,
\end{multline}
\begin{multline}
	A = \bbmat[c|cccc|c]
		-A_0 & \I & & & & \\
		& -A_1 & \I & & & \\
		& & \ddots & \ddots & & \\
		& & & -A_{N-2} & \I & \\
		& & & & -A_{N-1} & \I
	\ebmat,\\
	\quad B = \bbmat B_0 \\ & \ddots \\ && B_{N-1} \ebmat,
	\quad b = \bbmat b_0 \\ \vdots \\ b_{N-1} \ebmat,
\end{multline}
\begin{multline}
	C = \bbmat C_0 \\ & \ddots \\ && C_{N-1} \ebmat,
	\quad D = \bbmat D_0 \\ & \ddots \\ && D_{N-1} \ebmat,\\
	\quad \underline{d} = \bbmat \underline{d}_0 \\ \vdots \\ \underline{d}_{N-1} \ebmat,
	\quad \overline{d} = \bbmat \overline{d}_0 \\ \vdots \\ \overline{d}_{N-1} \ebmat,
\end{multline}
\begin{equation}
	x = \bbmat x_0 \\ \vdots \\ x_{N-1} \ebmat,
	\quad \underline{x} = \bbmat \underline{x}_0 \\ \vdots \\ \underline{x}_{N-1} \ebmat,
	\quad \overline{x} = \bbmat \overline{x}_0 \\ \vdots \\ \overline{x}_{N-1} \ebmat,
\end{equation}
\begin{equation}
	u = \bbmat u_0 \\ \vdots \\ u_{N-1} \ebmat,\\	
	\quad \underline{u} = \bbmat \underline{u}_0 \\ \vdots \\ \underline{u}_{N-1} \ebmat,
	\quad \overline{u} = \bbmat \overline{u}_0 \\ \vdots \\ \overline{u}_{N-1} \ebmat.
\end{equation}

\subsection{Condensing}
Using the equality constraints from \eqref{eq:OCP_QP}, $x$ can be expressed as a function of $x_0$ and $u$:
\begin{equation}
	x = F x_0 + G u + g.
\end{equation}
By substituting this expression into the objective function in \eqref{eq:OCP_QP_matrix_form}, we obtain a new objective with reduced number of variables:
\begin{multline}
	J(x_0,u) = \left(\bbmat F & G & g \\ & \I \\ && 1 \ebmat \bbmat x_0 \\ u \\ 1 \ebmat\right)^\top 
	\bbmat Q & S & q \\ S^\top & R & r \\ q^\top & r^\top & 0 \ebmat
	\bbmat F & G & g \\ & \I \\ && 1 \ebmat \bbmat x_0 \\ u \\ 1 \ebmat\\
	= \bbmat x_0 \\ u \\ 1 \ebmat^\top 
	\bbmat F^\top \\ G^\top & \I \\ g^\top && 1 \ebmat
	\bbmat Q & S & q \\ S^\top & R & r \\ q^\top & r^\top & 0 \ebmat
	\bbmat F & G & g \\ & \I \\ && 1 \ebmat 
	\bbmat x_0 \\ u \\ 1 \ebmat\\
	= \bbmat x_0 \\ u \\ 1 \ebmat^\top 
	\bbmat 
		F^\top QF & F^\top (QG + S) & F^\top(Qg + q) \\ 
		(QG + S)^\top F & G^\top Q G + S^\top G + G^\top S + R & G^\top (Qg + q) + S^\top g + r \\ 
		(Qg + q)^\top F & (Qg + q)^\top G + g^\top S + r^\top & g^\top Q g + q^\top g + g^\top q 
	\ebmat
	\bbmat x_0 \\ u \\ 1 \ebmat\ .
\end{multline}
Substitution of \eqref{} into \eqref{} and \eqref{} results in
\begin{equation}
	\bbmat \underline{x} \\ \underline{u} \ebmat
	\leq \bbmat F x_0 + G u + g \\ u \ebmat
	\leq \bbmat \overline{x} \\ \overline{u} \ebmat
\end{equation}
\begin{equation}
	\underline{d} \leq \bbmat C & D \ebmat\bbmat F x_0 + G u + g \\ u \ebmat \leq \overline{d}
\end{equation}
which gives the new box constraints
\begin{equation}
	\underline{u} \leq u \leq \overline{u}
\end{equation}
and new polytopic constraints
\begin{equation}	 
	\bbmat \underline{x} - g \\ \underline{d} - C g \ebmat
	\leq \bbmat F & G \\ C F & C G + D \ebmat \bbmat x_0 \\ u \ebmat 
	\leq \bbmat \overline{x} - g \\ \overline{d} - C g \ebmat \ .
\end{equation}
Since $F_0=\I$, $G_0=\ZeroMat$, $g_0=0$, the first rows of the polytopic constraints are actually box constraints:
\begin{equation}
	\bbmat \underline{x}_0 \\ \underline{u} \ebmat
	\leq \bbmat x_0 \\ u \ebmat 
	\leq \bbmat \overline{x}_0 \\ \overline{u} \ebmat
\end{equation}
\begin{equation}	 
	\bbmat \underline{x}_{1:N-1} - g_{1:N-1} \\ \underline{d} - C g \ebmat
	\leq \bbmat F_{1:N-1} & G_{1:N-1} \\ C F & C G + D \ebmat \bbmat x_0 \\ u \ebmat 
	\leq \bbmat \overline{x}_{1:N-1} - g_{1:N-1} \\ \overline{d} - C g \ebmat \ .
\end{equation}
We have transformed the original problem \eqref{eq:OCP_QP_matrix_form} to a new one with optimization variables $(x_0,u)$.

\bibliographystyle{plain} 

\end{document}